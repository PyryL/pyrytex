\documentclass[11pt,a4paper]{article}
\usepackage{pyrytex}
\title{Example usage of PyryTex}
\author{Pyry Lahtinen}
\date{2022-11-09}
\begin{document}

\maketitle

\section{Abbreviations}
PyryTex includes numerous abbreviations for frequently used characters.

\begin{figure}[h]
    \centering\begin{tabular}{c|c}
        Command & Output \\
        \hline
        \verb!\R! & $\R$ \\
        \verb!\N! & $\N$ \\
        \verb!\C! & $\C$ \\
        \verb!\Q! & $\Q$ \\
        \verb!\Z! & $\Z$ \\
        \verb!\Zp! & $\Zp$ \\
        \verb!\vv! & $\vv$ \\
        \verb!\vu! & $\vu$ \\
        \verb!\vw! & $\vw$ \\
        \verb!\vz! & $\vz$ \\
        \verb!\vi! & $\vi$ \\
        \verb!\vj! & $\vj$ \\
        \verb!\vk! & $\vk$ \\
        \verb!\ve! & $\ve$ \\
        \verb!\f{1}{2}! & $\f12$ \\
        \verb!\df{1}{2}! & $\df12$
    \end{tabular}
\end{figure}

\section{Included packages}
The following packages are imported automatically by PyryTex:
amsmath, amssymb, amsthm, geometry, units, xifthen, subfiles, graphicx, enumitem, listings

\section{Marking something as a work in progress}
Use the command \verb!\todo! to achieve this behaviour.

\todo

\section{Matrices}
Vertical lines can easily be added to matrices with a custom bracket parameter:

\begin{minipage}{0.5\textwidth}
    \begin{verbatim}
        \begin{bmatrix}[cc|c]
            8 & 9 & 12 \\
            5 & 2 & -2
        \end{bmatrix}
    \end{verbatim}
\end{minipage}%
\hfill
\begin{minipage}{0.5\textwidth}
    \begin{equation*}
        \begin{bmatrix}[cc|c]
            8 & 9 & 12 \\
            5 & 2 & -2
        \end{bmatrix}
    \end{equation*}
\end{minipage}

\end{document}